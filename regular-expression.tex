% WisdomFusion#gmail.com
% 20120602
%
\documentclass[12pt,a4paper,twoside]{ctexart}
%-------------------- 引用宏包--------------------
\usepackage[landscape,textheight=190mm,textwidth=145mm,left=20mm,right=20mm,top=20mm,bottom=15mm]{geometry}
\usepackage{fontspec}
\usepackage{fancybox}
\usepackage{colortbl}
\usepackage{fancyhdr}
\usepackage[CJKbookmarks,colorlinks,bookmarksnumbered=true,pdfstartview=FitH,linkcolor=black]{hyperref}
\usepackage{tabularx}
\usepackage{longtable}          % 自动换行用 tabularx,自动换页用 longtable,二者兼得用 ltxtable
\usepackage{multirow}
\usepackage{xcolor}
\usepackage{booktabs}
\usepackage{multicol}
\usepackage{lipsum}
\usepackage{menukeys}
%--------------------设定样式--------------------
\ctexset{
  section/format+=\raggedright
}

\usepackage{listings}
\lstset{
  upquote,
  keepspaces=true,
  columns=spaceflexible,
  numbers=left,
  basicstyle=\ttfamily\footnotesize,
  numberstyle=\ttfamily\tiny,
  keywordstyle=\color{blue}\ttfamily,
  stringstyle=\color{red}\ttfamily,
  commentstyle=\color{teal}\ttfamily,
  emphstyle=\color{blue}\bfseries,
  backgroundcolor=\color{yellow!5},
  frameround=ftft,
  frame=trBL
}
\setlength\columnsep{20pt}        % multicolumns 栏间距
\setlength{\tabcolsep}{0.5em}     % 水平padding
\renewcommand{\arraystretch}{1.5} % 垂直padding

% --------------------设定字体--------------------
%\setmainfont{Times New Roman}   % 缺省字体
\setmainfont{Georgia}   % 缺省字体
\setCJKfamilyfont{song}{SimSun}
\setCJKfamilyfont{hei}{SimHei}
\setCJKfamilyfont{kai}{KaiTi}
\setCJKfamilyfont{fs}{FangSong}
\setCJKfamilyfont{yahei}{Microsoft YaHei}

\renewcommand\contentsname{目\qquad{} 录}

%--------------------新建命令--------------------
\newcommand{\cbregex}[1]{\colorbox{orange!18}{\strut #1}}
\newcommand{\cbmatch}[1]{\colorbox{cyan!35}{\strut #1}}
\newcommand{\cbstring}[1]{\colorbox{green!20}{\strut #1}}

\begin{document}

\title{正则表达式快速参考手册}
\author{胡志飞 <WisdomFusion\#gmail.com>}
\date{2012年6月2日}

\maketitle{}
\thispagestyle{empty}
\clearpage{}

\tableofcontents{}
\thispagestyle{empty}
\clearpage{}

\setcounter{page}{1}

\section[简介]{简介 \textcolor{lightgray}{\textsc{Introduction}}}
\label{sec:intro}

正则表达式,(Regular Expression,在代码中常简写为regex、regexp或RE),计算机科学的一个概念。正则表达式使用字符来描述、匹配一系列符合某个句法规则的字符串。在很多文本编辑器里,正则表达式通常被用来检索、替换那些符合某个模式的文本。许多程序设计语言都支持利用正则表达式进行字符串操作。例如,在Perl\footnote{\href{https://www.perl.org/}{Perl}被称为“实用报表提取语言”(Practical Extraction and Report Language),正则表达式特性的推动者,文本处理非常方便。}中就内建了一个功能强大的正则表达式引擎。正则表达式这个概念最初是由Unix中的工具软件(例如sed\footnote{\href{http://www.gnu.org/software/sed/manual/sed.html}{sed}是一种UNIX/Linux平台下的轻量级流编辑器,日常一般用于处理文本文件。}和grep\footnote{grep,global search regular expression and print out the line,是一种强大的文本搜索工具,它能使用正则表达式搜索文本,并把匹配的行打印出来。})普及开的。\par

需要注意的是,用什么工具,用什么编辑语言,正则表达式的语法有些差别,特性的支持也参差不齐,称之为正则表达式“流派”(第\ref{sec:flavor}部分详述),所以要单独参考工具和编程语言本身的文档才行。\par

\section[基本语法]{基本语法 \textcolor{lightgray}{\textsc{Basic Syntax}}}
\label{sec:basic-syntax}

\begin{center}
  
\begin{longtable}{p{4em}p{8em}p{28em}p{16em}}
  \toprule
  \textbf{特性} & \textbf{语法} & \textbf{描述} & \textbf{举个栗子} \\
  \midrule
  \endfirsthead                 %
  \multicolumn{4}{l}{(续表)} \\
  \toprule
  \textbf{特性} & \textbf{语法} & \textbf{描述} & \textbf{举个栗子} \\
  \midrule
  \endhead                      %
  \midrule
  \multicolumn{4}{c}{To be continued\ldots} \\[2ex]
  \endfoot                      %
  \bottomrule
  \endlastfoot                  %
  %--------------------
  字符 & 除[\textbackslash{}\^{}\$.|?*+()以外的任意字符 & 除了[\textbackslash{}\^{}\$.|?*+()以外的任意字符,\{和\}也是文字文本,除了下面说到的成对出现的量词语法,如\{n\}和\{m,n\}等。& \cbregex{a}匹配\cbstring{about}中的\cbmatch{a} \\
  & 字符转义 & \cbregex{\textbackslash{}t}, \cbregex{\textbackslash{}?}, \cbregex{\textbackslash{}*}, \cbregex{\textbackslash{}+}, \cbregex{\textbackslash{}.}, \cbregex{\textbackslash{}|}, \cbregex{\textbackslash{}\{}, \cbregex{\textbackslash{}\}}, \cbregex{\textbackslash\textbackslash}, \cbregex{\textbackslash{}[}, \cbregex{\textbackslash{}]}, \cbregex{\textbackslash{}(}, \cbregex{\textbackslash{})} & \cbregex{\textbackslash{}+}匹配\cbmatch{+};\cbregex{\textbackslash{}?\textbackslash{}-}匹配\cbmatch{?-} \\
  & \cbregex{\textbackslash{}n}, \cbregex{\textbackslash{}r} 和 \cbregex{\textbackslash{}t} & Windows文件格式换行符是\textbackslash{}r\textbackslash{}n,UNIX文件格式换行符是\textbackslash{}n,\textbackslash{}t 匹配水平制表符 & \\
  & \cbregex{\textbackslash{}cA}到\cbregex{\textbackslash{}cZ}, \cbregex{\textbackslash{}ca}到\cbregex{\textbackslash{}cz} & \keys{Ctrl}+\keys{A} 到 \keys{Ctrl}+\keys{Z},与ASCII字符\cbregex{\textbackslash{}x01}到\cbregex{\textbackslash{}x1A}等价 & \\
  & \cbregex{\textbackslash{}a}, \cbregex{\textbackslash{e}}, \cbregex{\textbackslash{}f}, \cbregex{\textbackslash{}v} & 依次为警报(\textbackslash{}x07)、Esc字符(\textbackslash{}x1B)、进纸符(\textbackslash{}x0C)和垂直制表符(\textbackslash{}x0B) & \\
  & \cbregex{\textbackslash{}Q}\ldots\cbregex{\textbackslash{}E} & 文字文本范围,被包含在\cbregex{\textbackslash{}Q}和\cbregex{\textbackslash{}E}之间的文字,都被视为普通文字,如[\textbackslash{}\^{}\$.|?*+()\{\}也不再用转义了,这个最早是由Perl引入正则表达式的。& \cbregex{\textbackslash{}Q+-*/\textbackslash{}E}匹配的就是\cbmatch{+-*/} \\
  \midrule
  %--------------------
  基本特性 & \cbregex{.}(点) & 匹配除换行符之外的任意字符,有些正则表达式“流派”还支持点是否匹配换行符的开关。& \cbregex{.} 匹配 about 中的任意一个字符 \\
  & \cbregex{|} & 管道,或的关系,匹配|的左侧或右侧的字符串 & \cbregex{abc|def|xyz}匹配\cbmatch{abc}或\cbmatch{def}或\cbmatch{xyz} \\
  \midrule
  %--------------------
  字符类 & \cbregex{[\ldots]} & 匹配字符类中列举的任意一个字符 & \cbregex{[abc]}匹配\cbmatch{a}或\cbmatch{b}或\cbmatch{c};\cbregex{[.!?]}匹配\cbmatch{.}或\cbmatch{!}或\cbmatch{?} \\
  & \cbregex{[\textbackslash{}\^{}\textbackslash{}]]} & 在字符类中,要匹配 \^{}-]\textbackslash{}这几字符,得使用\textbackslash{}转义 & \cbregex{[\textbackslash{}\^{}\textbackslash{}]]}匹配\cbmatch{\^{}}或\cbmatch{]} \\
  & \cbregex{[\^{}\ldots]} & 排除型字符类,\^{}(脱字符,caret)紧跟 [ 之后,可以把字符类中列举的字符排除匹配范围,也就是所这个字符类将匹配任意一个不在列出字符范围内的字符 & \cbregex{[\^{}a-d]}匹配除了a,b,c,d之外的任意一个字符 \\
  & \cbregex{\textbackslash{}d}, \cbregex{\textbackslash{}w}, \cbregex{\textbackslash{}s} & \cbregex{\textbackslash{}d} 匹配数字,与 \cbregex{[0-9]} 等价;\cbregex{\textbackslash{}w} 匹配任意一个字母或数字或下划线或汉字;\cbregex{\textbackslash{}s} 匹配任意一个空白符 & \cbregex{[\textbackslash{}d\textbackslash{}s]}匹配一个数字或空白符 \\
  & \cbregex{\textbackslash{}D}, \cbregex{\textbackslash{}W}, \cbregex{\textbackslash{}S} & 是 \cbregex{\textbackslash{}d}, \cbregex{\textbackslash{}w} 和 \cbregex{\textbackslash{}s} 的反义字符类。\cbregex{\textbackslash{}D} 匹配任意非数字的字符;\cbregex{\textbackslash{}W} 匹配任意不是字母、数字、下划线、汉字的字符;\cbregex{\textbackslash{}S} 匹配任意不是空白符的字符 & \textbackslash{}D匹配任意非数字的字符 \\
  & \cbregex{[\textbackslash{}b]} & 在字符类中,\cbregex{[\textbackslash{}b]}为\keys{Backspace}退格键字符 & \\
  \midrule
  %--------------------
  POSIX & \cbregex{[:alnum:]} & & \\
  & \cbregex{[:alpha:]} & & \\
  & \cbregex{[:ascii:]} & & \\
  & \cbregex{[:blank:]} & & \\
  & \cbregex{[:cntrl:]} & & \\
  & \cbregex{[:digit:]} & & \\
  & \cbregex{[:graph:]} & & \\
  & \cbregex{[:lower:]} & & \\
  & \cbregex{[:print:]} & & \\
  & \cbregex{[:punct:]} & & \\
  & \cbregex{[:space:]} & & \\
  & \cbregex{[:upper:]} & & \\
  & \cbregex{[:word:]} & & \\
  & \cbregex{[:xdigit:]} & & \\
  \midrule
  %--------------------
  锚点 & \cbregex{\^{}} & & \\
  & \cbregex{\$} & & \\
  & \cbregex{\textbackslash{}A} & & \\
  & \cbregex{\textbackslash{}G} & & \\
  & \cbregex{\textbackslash{}Z} & & \\
  & \cbregex{\textbackslash{}b} & & \\
  & \cbregex{\textbackslash{}B} & & \\
  & \cbregex{\textbackslash{}<} & & \\
  & \cbregex{\textbackslash{}>} & & \\
  & \cbregex{\textbackslash{}\`{}} & & \\
  & \cbregex{\textbackslash{}\&} & & \\
  & \cbregex{\textbackslash{}'} & & \\
  & & & \\
  \midrule
  %--------------------
  量词 & \cbregex{?} & \\
  & \cbregex{??} & & \\
  & \cbregex{?+} & & \\
  & \cbregex{*} & & \\
  & \cbregex{*?} & & \\
  & \cbregex{*+} & & \\
  & \cbregex{+} & & \\
  & \cbregex{+?} & \\
  & \cbregex{++} & & \\
  & \cbregex{\{n\}} & & \\
  & \cbregex{\{n,m\}} & & \\
  & \cbregex{\{n,\}} & & \\
  & \cbregex{\{,m\}} & & \\
  & \cbregex{\{n,m\}?} & & \\
  & \cbregex{\{n,m\}+} & & \\
\end{longtable}
\end{center}

\section[高级语法]{高级语法 \textcolor{lightgray}{\textsc{Advanced Syntax}}}
\label{sec:adv-syntax}

\begin{center}

\begin{longtable}{p{4em}p{9em}p{28em}p{15em}}
  \toprule
  \textbf{特性} & \textbf{语法} & \textbf{描述} & \textbf{举个栗子} \\
  \midrule
  \endfirsthead                 %
  \multicolumn{4}{l}{(续表)} \\
  \toprule
  \textbf{特性} & \textbf{语法} & \textbf{描述} & \textbf{举个栗子} \\
  \midrule
  \endhead                      %
  \midrule
  \multicolumn{4}{c}{To be continued\ldots} \\[2ex]
  \endfoot                      %
  \bottomrule
  \endlastfoot                  %
  %--------------------
  Unicode & & & \\
  & & & \\
  & & & \\
  & & & \\
  & & & \\
  & & & \\
  & & & \\
  & & & \\
  \midrule
  %--------------------
  \multirow{2}{4em}{分组与反向引用} & (regex) & & \\
  & (?:regex) & & \\
  & \textbackslash{}1 到 \textbackslash{}9 & & \\
  & \textbackslash{}10 到 \textbackslash{}99 & & \\
  & \textbackslash{}g\{1\} 到 \textbackslash{}g\{99\} & & \\
  & \textbackslash{}g\{-1\}, \textbackslash{}g\{-2\}, etc. & & \\
  & (?<name>regex) & & \\
  & \textbackslash{}k<name>, \textbackslash{}g\{name\} & & \\
  \midrule
  模式修饰符 & \cbregex{(?i)} & & \\
  & \cbregex{(?-i)} & & \\
  & \cbregex{(?s)} & & \\
  & \cbregex{(?-s)} & & \\
  & \cbregex{(?m)} & & \\
  & \cbregex{(?-m)} & & \\
  & \cbregex{(?x)} & & \\
  
  %--------------------
  高级分组 & (?\#comment) & & \\
  & (?=Regex) & & \\
  & (?!Regex) & & \\
  & (?<=regex) & & \\
  & (?<!regex) & & \\
\end{longtable}

\end{center}

\section[举些栗子]{举些栗子 \textcolor{lightgray}{\textsc{Regex Examples}}}
\label{sec:regex-examples}

\begin{multicols}{2}
一些栗子 \par


\end{multicols}

\section[正则表达式“流派”]{正则表达式“流派” \textcolor{lightgray}{\textsc{Regex Flavors}}}
\label{sec:flavor}

\begin{multicols}{2}
流派

\end{multicols}


\section[应用场景]{应用场景 \textcolor{lightgray}{\textsc{Application Scenarios}}}
\label{sec:scenarios}

\subsection[正则表达式工具箱]{正则表达式工具箱 \textcolor{lightgray}{\textsc{Regex Toolbox}}}
\label{sec:toolbox}

总有一款适合你,Windows下的记事本太鸡肋,Word处理方式主要是“通配符”而不是正则表达式。 \par

\begin{multicols}{2}
  
\noindent\textbf{RegexBuddy} \par
JGsoft开发的一个强大的正则表达式测试工具, \par

\noindent\textbf{grep}

\noindent\textbf{PowerGREP} \par
RegexBuddy的兄弟软件,同是JGsoft开发,是grep在Windows平台的实现和增强。

\noindent\textbf{UltraEdit, Notepad++}

\noindent\textbf{Vim} \par

\noindent\textbf{GNU Emacs} \par

\noindent\textbf{sed \& awk}

\end{multicols}

\subsection[应用案例]{应用案例 \textcolor{lightgray}{\textsc{Application Cases}}}
\label{sec:cases}

\noindent\textbf{Dreamweaver 表格处理} \par

\noindent\textbf{VBA 中使用正则表达式} \par

\begin{lstlisting}[language=VBScript]
Sub IndentParaWithRegEx()
' PowerPoint VBA 批量给指定字符开头段落加动画
Dim oSld As Slide
Dim oShp As Shape
Dim i As Integer
' 正则相关变量
Dim regx As Object, oMatch As Object

' 这里写查找的正则,参考 http://msdn.microsoft.com/en-us/library/ms974570.aspx
strPattern = "^开头字符串"

Set regx = CreateObject("vbscript.regexp")
With regx
    .Global = True
    .IgnoreCase = True
    .Pattern = strPattern
End With
    
For Each oSld In ActivePresentation.Slides
    For Each oShp In oSld.Shapes
        If oShp.HasTextFrame Then
            If oShp.TextFrame2.HasText Then
                With oShp.TextFrame2.TextRange
                    For i = 1 To .Paragraphs.Count
                        With .Paragraphs(i)
                             ' 可能会出现多个匹配项的
                            If (regx.Test(.Text) = True) Then
                                .ParagraphFormat.FirstLineIndent = 0
                            End If
                        End With
                    Next i 'para
                End With
            End If 'has text
        End If 'has textframe
    Next oShp
Next oSld
End Sub  
\end{lstlisting}

\noindent\textbf{InDesign GREP} \par

\noindent\textbf{使用Perl正则表达式处理文件} \par

\noindent\textbf{神的编辑器之正则} \par

\end{document}

%%% Local Variables:
%%% mode: latex
%%% TeX-master: t
%%% End:
