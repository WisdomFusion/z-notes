\documentclass[12pt,a4paper,twoside]{ctexart}
%
\usepackage[landscape,textheight=190mm,textwidth=145mm,left=20mm,right=20mm,top=20mm,bottom=20mm]{geometry}
\usepackage{fontspec}
\usepackage{fancybox}
\usepackage{colortbl}
\usepackage{fancyhdr}
\usepackage[CJKbookmarks, colorlinks,bookmarksnumbered=true,pdfstartview=FitH,linkcolor=black]{hyperref}
\usepackage{tabularx}
\usepackage{longtable}          % 自动换行用 tabularx,自动换页用 longtable,二者兼得用 ltxtable
\usepackage{multirow}
\usepackage{xcolor}
\usepackage{booktabs}
\usepackage{lipsum}
%--------------------
\setmainfont{Times New Roman}   % 缺省字体
\setCJKfamilyfont{song}{SimSun}
\setCJKfamilyfont{hei}{SimHei}
\setCJKfamilyfont{kai}{KaiTi}
\setCJKfamilyfont{fs}{FangSong}
\setCJKfamilyfont{yahei}{Microsoft YaHei}

\renewcommand\contentsname{目\qquad{} 录}
\setlength{\tabcolsep}{0.5em}
\renewcommand{\arraystretch}{1.5}
%--------------------
\colorlet{regexcolor}{orange!15}
\colorlet{matchcolor}{cyan!50}
\colorlet{stringcolor}{green!20}

\newcommand{\cbregex}[1]{\colorbox{orange!30}{\strut #1}}
\newcommand{\cbmatch}[1]{\colorbox{cyan!50}{\strut #1}}
\newcommand{\cbstring}[1]{\colorbox{green!30}{\strut #1}}

\begin{document}

\title{正则表达式快速参考手册}
\author{胡志飞 <WisdomFusion\#gmail.com>}
\maketitle{}
\thispagestyle{empty}
\clearpage{}

\tableofcontents{}
\thispagestyle{empty}
\clearpage{}

\setcounter{page}{1}

\section[简介]{简介 Introduction}
\label{sec:intro}

\lipsum[1]

\lipsum[2]

\section[基本语法]{基本语法 Basic Syntax}
\label{sec:basic-syntax}


\centering
\noindent
\begin{longtable}{p{4em}p{9em}p{28em}p{15em}}
  \toprule
  \textbf{特性} & \textbf{语法} & \textbf{描述} & \textbf{举个栗子} \\
  \midrule
  \endfirsthead                 %
  \multicolumn{4}{l}{(续表)} \\
  \toprule
  \textbf{特性} & \textbf{语法} & \textbf{描述} & \textbf{举个栗子} \\
  \midrule
  \endhead                      %
  \midrule
  \multicolumn{4}{c}{To be continued...} \\[2ex]
  \endfoot                      %
  \bottomrule
  \endlastfoot                  %
  %--------------------
  字符 & 除[\textbackslash{}\^{}\$.|?*+()以外的任意字符 & 除了[\textbackslash{}\^{}\$.|?*+()以外的任意字符,{和}也是文字文本,除了下面说到的成对出现的量词语法,如{n}和{m,n}等。& \cbregex{a}匹配\cbstring{about}中的\cbmatch{a} \\
  & ddd & ddd & dddd \\
  & \cbregex{\textbackslash{}n}, \cbregex{\textbackslash{}r} & & \\
  & \cbregex{\textbackslash{}cA}到\cbregex{\textbackslash{}cZ}, \cbregex{\textbackslash{}ca}到\cbregex{\textbackslash{}cz} & & \\
  & \cbregex{\textbackslash{}a} & & \\
  & \cbregex{\textbackslash{}f} & & \\
  & \cbregex{\textbackslash{}v} & & \\
  \midrule
  %--------------------
  点 & \cbregex{.}(点) & Matches any single character except line break characters. Most regex flavors have an option to make the dot match line break characters too. & . 匹配 about 中的任意一个字符 \\
  管道 & \cbregex{|} & & \\
  \midrule
  %--------------------
  字符类 & \cbregex{[abc]}, \cbregex{[\textbackslash{}\^{}\textbackslash{}]]} & & \\
  & \cbregex{\textbackslash{}d}, \cbregex{\textbackslash{}D} & & \\
  & \cbregex{\textbackslash{}w}, \cbregex{\textbackslash{}W} & & \\
  & \cbregex{\textbackslash{}s}, \cbregex{\textbackslash{}S} & & \\
  \midrule
  %--------------------
  POSIX & \cbregex{[:alnum:]} & & \\
  & \cbregex{[:alpha:]} & & \\
  & \cbregex{[:ascii:]} & & \\
  & \cbregex{[:blank:]} & & \\
  & \cbregex{[:cntrl:]} & & \\
  & \cbregex{[:digit:]} & & \\
  & \cbregex{[:graph:]} & & \\
  & \cbregex{[:lower:]} & & \\
  & \cbregex{[:print:]} & & \\
  & \cbregex{[:punct:]} & & \\
  & \cbregex{[:space:]} & & \\
  & \cbregex{[:upper:]} & & \\
  & \cbregex{[:word:]} & & \\
  & \cbregex{[:xdigit:]} & & \\
  \midrule
  %--------------------
  锚点 & \cbregex{\^{}} & & \\
  & \cbregex{\$} & & \\
  & \cbregex{\textbackslash{}A} & & \\
  & \cbregex{\textbackslash{}G} & & \\
  & \cbregex{\textbackslash{}Z} & & \\
  & \cbregex{\textbackslash{}b} & & \\
  & \cbregex{\textbackslash{}B} & & \\
  & \cbregex{\textbackslash{}<} & & \\
  & \cbregex{\textbackslash{}>} & & \\
  & \cbregex{\textbackslash{}`} & & \\
  & \cbregex{\textbackslash{}\&} & & \\
  & \cbregex{\textbackslash{}'} & & \\
  \midrule
  %--------------------
  量词 & ? & \\
  & 2 & & \\
  & 3 & & \\
  & 4 & & \\
  & 5 & & \\
  & 6 & & \\
  & 7 & & \\
  & 8 & \\
  & 9 & & \\
  & 10 & & \\
  & 11 & & \\
  & 12 & & \\
  & 13 & & \\
  & 14 & & \\
  
\end{longtable}


\section[高级语法]{高级语法 Advanced Syntax}
\label{sec:adv-syntax}


\section[正则表达式“流派”]{正则表达式“流派” Regex Flavors}
\label{sec:flavor}


\section[应用场景]{应用场景 Application Scenarios}
\label{sec:scenarios}


\end{document}

%%% Local Variables:
%%% mode: latex
%%% TeX-master: t
%%% End:
