\documentclass[12pt,a4paper,twoside]{ctexart}
\usepackage[landscape,textheight=190mm,textwidth=145mm,left=20mm,right=20mm,top=20mm,bottom=20mm]{geometry}
\usepackage{fontspec}
\usepackage{fancybox}
\usepackage{colortbl}
\usepackage{fancyhdr}
\usepackage[CJKbookmarks, colorlinks,bookmarksnumbered=true,pdfstartview=FitH,linkcolor=black]{hyperref}
\usepackage{tabularx}
\usepackage{longtable}          % 自动换行用 tabularx,自动换页用 longtable,二者兼得用 ltxtable
\usepackage{multirow}
\usepackage{xcolor}
\usepackage{booktabs}
\usepackage{multicol}
\usepackage{lipsum}

\usepackage{listings}
\lstset{
  upquote,
  keepspaces=true,
  columns=spaceflexible,
  numbers=left,
  basicstyle=\ttfamily\footnotesize,
  numberstyle=\ttfamily\tiny,
  keywordstyle=\color{blue}\ttfamily,
  stringstyle=\color{red}\ttfamily,
  commentstyle=\color{green}\ttfamily,
  commentstyle=\sffamily\itshape\smaller\color{SeaGreen4},
  emphstyle=\bfseries\color{RoyalBlue3},
  emphstyle={[2]{\bfseries\color{Sienna2}}},
  backgroundcolor=\color{yellow!5},
  frameround=ftft,
  frame=trBL
}

%--------------------
%\setmainfont{Times New Roman}   % 缺省字体
\setmainfont{Georgia}   % 缺省字体
\setCJKfamilyfont{song}{SimSun}
\setCJKfamilyfont{hei}{SimHei}
\setCJKfamilyfont{kai}{KaiTi}
\setCJKfamilyfont{fs}{FangSong}
\setCJKfamilyfont{yahei}{Microsoft YaHei}

\ctexset{
  section/format+=\raggedright
}

\renewcommand\contentsname{目\qquad{} 录}

\setlength\columnsep{20pt}        % multicolumns 栏间距

\setlength{\tabcolsep}{0.5em}     % 水平padding
\renewcommand{\arraystretch}{1.5} % 垂直padding
%--------------------
\colorlet{regexcolor}{orange!15}
\colorlet{matchcolor}{cyan!50}
\colorlet{stringcolor}{green!20}

\newcommand{\cbregex}[1]{\colorbox{orange!30}{\strut #1}}
\newcommand{\cbmatch}[1]{\colorbox{cyan!50}{\strut #1}}
\newcommand{\cbstring}[1]{\colorbox{green!30}{\strut #1}}

\begin{document}

\title{正则表达式快速参考手册}
\author{胡志飞 <WisdomFusion\#gmail.com>}
\maketitle{}
\thispagestyle{empty}
\clearpage{}

\tableofcontents{}
\thispagestyle{empty}
\clearpage{}

\setcounter{page}{1}

\section[简介]{简介 \textcolor{lightgray}{\textsc{Introduction}}}
\label{sec:intro}

中文文字 \par
中文文字 \par
\lipsum[1] \par
\lipsum[2] \par

\section[基本语法]{基本语法 \textcolor{lightgray}{\textsc{Basic Syntax}}}
\label{sec:basic-syntax}

\begin{center}
  
\begin{longtable}{p{4em}p{9em}p{28em}p{15em}}
  \toprule
  \textbf{特性} & \textbf{语法} & \textbf{描述} & \textbf{举个栗子} \\
  \midrule
  \endfirsthead                 %
  \multicolumn{4}{l}{(续表)} \\
  \toprule
  \textbf{特性} & \textbf{语法} & \textbf{描述} & \textbf{举个栗子} \\
  \midrule
  \endhead                      %
  \midrule
  \multicolumn{4}{c}{To be continued\ldots} \\[2ex]
  \endfoot                      %
  \bottomrule
  \endlastfoot                  %
  %--------------------
  字符 & 除[\textbackslash{}\^{}\$.|?*+()以外的任意字符 & 除了[\textbackslash{}\^{}\$.|?*+()以外的任意字符,{和}也是文字文本,除了下面说到的成对出现的量词语法,如{n}和{m,n}等。& \cbregex{a}匹配\cbstring{about}中的\cbmatch{a} \\
  & 转义 & \textbackslash{}t, \textbackslash{}?, \textbackslash{}*, \textbackslash{}+, \textbackslash{}., \textbackslash{}|, \textbackslash{}\{, \textbackslash{}\}, \textbackslash{}\textbackslash{}, \textbackslash{}[, \textbackslash{}], \textbackslash{}(, \textbackslash{}) & \\
  & \cbregex{\textbackslash{}n}, \cbregex{\textbackslash{}r} & & \\
  & \cbregex{\textbackslash{}cA}到\cbregex{\textbackslash{}cZ}, \cbregex{\textbackslash{}ca}到\cbregex{\textbackslash{}cz} & & \\
  & \cbregex{\textbackslash{}a} & & \\
  & \cbregex{\textbackslash{}f} & & \\
  & \cbregex{\textbackslash{}v} & & \\
  \midrule
  %--------------------
  基本特性 & \cbregex{.}(点) & Matches any single character except line break characters. Most regex flavors have an option to make the dot match line break characters too. & . 匹配 about 中的任意一个字符 \\
  & \cbregex{|} & & \\
  \midrule
  %--------------------
  字符类 & \cbregex{[abc]}, \cbregex{[\textbackslash{}\^{}\textbackslash{}]]} & & \\
  & \cbregex{[\^{}abc]]} & & \\
  & \cbregex{\textbackslash{}d}, \cbregex{\textbackslash{}D} & & \\
  & \cbregex{\textbackslash{}w}, \cbregex{\textbackslash{}W} & & \\
  & \cbregex{\textbackslash{}s}, \cbregex{\textbackslash{}S} & & \\
  \midrule
  %--------------------
  POSIX & \cbregex{[:alnum:]} & & \\
  & \cbregex{[:alpha:]} & & \\
  & \cbregex{[:ascii:]} & & \\
  & \cbregex{[:blank:]} & & \\
  & \cbregex{[:cntrl:]} & & \\
  & \cbregex{[:digit:]} & & \\
  & \cbregex{[:graph:]} & & \\
  & \cbregex{[:lower:]} & & \\
  & \cbregex{[:print:]} & & \\
  & \cbregex{[:punct:]} & & \\
  & \cbregex{[:space:]} & & \\
  & \cbregex{[:upper:]} & & \\
  & \cbregex{[:word:]} & & \\
  & \cbregex{[:xdigit:]} & & \\
  \midrule
  %--------------------
  锚点 & \cbregex{\^{}} & & \\
  & \cbregex{\$} & & \\
  & \cbregex{\textbackslash{}A} & & \\
  & \cbregex{\textbackslash{}G} & & \\
  & \cbregex{\textbackslash{}Z} & & \\
  & \cbregex{\textbackslash{}b} & & \\
  & \cbregex{\textbackslash{}B} & & \\
  & \cbregex{\textbackslash{}<} & & \\
  & \cbregex{\textbackslash{}>} & & \\
  & \cbregex{\textbackslash{}\`{}} & & \\
  & \cbregex{\textbackslash{}\&} & & \\
  & \cbregex{\textbackslash{}'} & & \\
  & & & \\
  \midrule
  %--------------------
  量词 & \cbregex{?} & \\
  & \cbregex{??} & & \\
  & \cbregex{?+} & & \\
  & \cbregex{*} & & \\
  & \cbregex{*?} & & \\
  & \cbregex{*+} & & \\
  & \cbregex{+} & & \\
  & \cbregex{+?} & \\
  & \cbregex{++} & & \\
  & \cbregex{\{n\}} & & \\
  & \cbregex{\{n,m\}} & & \\
  & \cbregex{\{n,\}} & & \\
  & \cbregex{\{,m\}} & & \\
  & \cbregex{\{n,m\}?} & & \\
  & \cbregex{\{n,m\}+} & & \\
\end{longtable}
\end{center}

\section[高级语法]{高级语法 \textcolor{lightgray}{\textsc{Advanced Syntax}}}
\label{sec:adv-syntax}

\begin{center}

\begin{longtable}{p{4em}p{9em}p{28em}p{15em}}
  \toprule
  \textbf{特性} & \textbf{语法} & \textbf{描述} & \textbf{举个栗子} \\
  \midrule
  \endfirsthead                 %
  \multicolumn{4}{l}{(续表)} \\
  \toprule
  \textbf{特性} & \textbf{语法} & \textbf{描述} & \textbf{举个栗子} \\
  \midrule
  \endhead                      %
  \midrule
  \multicolumn{4}{c}{To be continued\ldots} \\[2ex]
  \endfoot                      %
  \bottomrule
  \endlastfoot                  %
  %--------------------
  Unicode & & & \\
  & & & \\
  & & & \\
  & & & \\
  & & & \\
  & & & \\
  & & & \\
  & & & \\
  \midrule
  %--------------------
  \multirow{2}{4em}{分组与反向引用} & (regex) & & \\
  & (?:regex) & & \\
  & \textbackslash{}1 到 \textbackslash{}9 & & \\
  & \textbackslash{}10 到 \textbackslash{}99 & & \\
  & \textbackslash{}g\{1\} 到 \textbackslash{}g\{99\} & & \\
  & \textbackslash{}g\{-1\}, \textbackslash{}g\{-2\}, etc. & & \\
  & (?<name>regex) & & \\
  & \textbackslash{}k<name>, \textbackslash{}g\{name\} & & \\
  \midrule
  %--------------------
  高级分组 & (?\#comment) & & \\
  & (?=Regex) & & \\
  & (?!Regex) & & \\
  & (?<=regex) & & \\
  & (?<!regex) & & \\
\end{longtable}

\end{center}

\section[举些栗子]{举些栗子 \textcolor{lightgray}{\textsc{Regex Examples}}}
\label{sec:regex-examples}

\begin{multicols}{2}
一些栗子 \par


\end{multicols}

\section[正则表达式“流派”]{正则表达式“流派” \textcolor{lightgray}{\textsc{Regex Flavors}}}
\label{sec:flavor}

\begin{multicols}{2}
流派

\end{multicols}


\section[应用场景]{应用场景 \textcolor{lightgray}{\textsc{Application Scenarios}}}
\label{sec:scenarios}

\subsection[工具箱]{工具箱 \textcolor{lightgray}{\textsc{Toolbox}}}
\label{sec:toolbox}

总有一款适合你,Windows下的记事本太鸡肋,Word处理方式主要是“通配符”而不是正则表达式。 \par

\begin{multicols}{2}
  
\noindent\textbf{RegexBuddy} \par
JGsoft开发的一个强大的正则表达式测试工具, \par

\noindent\textbf{grep}

\noindent\textbf{PowerGREP} \par
RegexBuddy的兄弟软件,同是JGsoft开发,是grep在Windows平台的实现和增强。

\noindent\textbf{UltraEdit, Notepad++}

\noindent\textbf{Vim} \par

\noindent\textbf{GNU Emacs} \par

\noindent\textbf{sed \& awk}

\end{multicols}

\subsection[应用案例]{应用案例 \textcolor{lightgray}{\textsc{Application Cases}}}
\label{sec:cases}

\noindent\textbf{Dreamweaver 表格处理} \par

\noindent\textbf{VBA 中使用正则表达式} \par

\begin{lstlisting}[language=VBScript]
Sub IndentParaWithRegEx()
' PowerPoint VBA 批量给指定字符开头段落加动画
Dim oSld As Slide
Dim oShp As Shape
Dim i As Integer
' 正则相关变量
Dim regx As Object, oMatch As Object

' 这里写查找的正则,参考 http://msdn.microsoft.com/en-us/library/ms974570.aspx
strPattern = "^开头字符串"

Set regx = CreateObject("vbscript.regexp")
With regx
    .Global = True
    .IgnoreCase = True
    .Pattern = strPattern
End With
    
For Each oSld In ActivePresentation.Slides
    For Each oShp In oSld.Shapes
        If oShp.HasTextFrame Then
            If oShp.TextFrame2.HasText Then
                With oShp.TextFrame2.TextRange
                    For i = 1 To .Paragraphs.Count
                        With .Paragraphs(i)
                             ' 可能会出现多个匹配项的
                            If (regx.Test(.Text) = True) Then
                                .ParagraphFormat.FirstLineIndent = 0
                            End If
                        End With
                    Next i 'para
                End With
            End If 'has text
        End If 'has textframe
    Next oShp
Next oSld
End Sub  
\end{lstlisting}

\noindent\textbf{InDesign GREP} \par

\noindent\textbf{使用Perl正则表达式处理文件} \par

\noindent\textbf{神的编辑器之正则} \par

\end{document}

%%% Local Variables:
%%% mode: latex
%%% TeX-master: t
%%% End:
